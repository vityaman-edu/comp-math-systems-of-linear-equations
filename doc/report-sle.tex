\documentclass{article}

\usepackage[utf8]{inputenc}
\usepackage[russian]{babel}
\usepackage[a4paper, margin=1in]{geometry}
\usepackage{graphicx}
\usepackage{amsmath}
\usepackage{wrapfig}
\usepackage{multirow}
\usepackage{mathtools}
\usepackage{pgfplots}
\usepackage{pgfplotstable}
\usepackage{setspace}
\usepackage{changepage}
\usepackage{caption}
\usepackage{csquotes}
\usepackage{hyperref}
\usepackage{listings}

\pgfplotsset{compat=1.18}
\hypersetup{
  colorlinks = true,
  linkcolor  = blue,
  filecolor  = magenta,      
  urlcolor   = darkgray,
  pdftitle   = {
    comp-math-lab-1-report-systems-of-linear-equations-smirnov-victor
  },
}

\definecolor{codegreen}{rgb}{0,0.6,0}
\definecolor{codegray}{rgb}{0.5,0.5,0.5}
\definecolor{codepurple}{rgb}{0.58,0,0.82}
\definecolor{backcolour}{rgb}{0.99,0.99,0.99}

\lstdefinestyle{mystyle}{
  backgroundcolor=\color{backcolour},   
  commentstyle=\color{codegreen},
  keywordstyle=\color{magenta},
  numberstyle=\tiny\color{codegray},
  stringstyle=\color{codepurple},
  basicstyle=\ttfamily\footnotesize,
  breakatwhitespace=false,         
  breaklines=true,                 
  captionpos=b,                    
  keepspaces=true,                 
  numbers=left,                    
  numbersep=5pt,                  
  showspaces=false,                
  showstringspaces=false,
  showtabs=false,                  
  tabsize=2
}

\lstset{style=mystyle}

\begin{document}

\begin{titlepage}
    \begin{center}
        \begin{spacing}{1.4}
            \large{Университет ИТМО} \\
        \end{spacing}
        \vfill
        \textbf{
            \huge{Лабораторная работа №1.} \\
            \huge{Решение системы линейных алгебраических уравнений}
        }
    \end{center}
    \vfill
    \begin{center}
        \begin{tabular}{r l}
            Группа:  & P32131                  \\
            Студент: & Смирнов Виктор Игоревич \\
            Вариант: & 2 (Метод итераций)      \\
        \end{tabular}
    \end{center}
    \vfill
    \begin{center}
        \begin{large}
            2023
        \end{large}
    \end{center}
\end{titlepage}

\section{Цель работы}
В новом семестре в рамках первой лабораторной работы
была поставлена задача реализовать так называемый метод
простых итераций для нахождения решения системы
линейных алгебраических уравнений (далее СЛУ).

\section{Задачи, решаемые при выполнении}
\begin{enumerate}
    \item Изучить метод итераций
    \item Разработать алгоритм приведения матрицы СЛУ
          эквивалентными преобразованиями к такому,
          в котором выполнено диагональное преобладание
    \item Разработать приложение для решения СЛУ
\end{enumerate}

\section{Описание метода итераций для решения СЛУ}
Сам алгоритм, условия сходимости и прочее прекрасно изложены
в книге Демидовича Б. П. и Марона И. А.
- "Основы вычислительной математики" 1966 года издания с
294 до 302 страниц, так что читатель может и должен для
дальнейшнего понимания происходящего ознакомиться с ним
самостоятельно.

\section{Получение СЛУ с диагональным преобладанием}
Одним из достаточных условий сходимости метода приближений является
наличие свойства диагонального преобладания матрицы коэффициентов
(далее будем называть ее основной матрицей).

\begin{equation}
    \forall i \in [1..n] \sum_{i \neq j} |a_{i, j}| < |a_{i,i}|
\end{equation}

Для приведения основной матрицы к нужному нам виду, нам
нужно применить к ней некоторое количество элементарных
преобразований. Напомню, что элементарные преобразования не
меняют решений СЛУ, так что мы не "испортим" нашу систему ими.

Один из методов также был описан в
"Основах вычислительной математики" на странице 301.
Будем следовать его описанию.

Вообще, самым простым элементарным преобразваением является
перестановка строк местами, так что давайте начнем с него.
Чего мы можем добиться, используя только это преобразование?
Можем максимально заполнить строки матрицы подходящими.

Пусть $P$ - множество позиций строк желаемой матрицы с
диагональным преобладанием, а $R$ - множество
строк изначальной матрицы.
Рассмотрим двудольный граф $G = (P, R, E): (p, r) \in E
    \Leftrightarrow ($для строки $r$ выполняется диагональное
    преобаладание на позиции $p)$. Заметим, что задача перестановки
строк в нашей матрице таким образом, чтобы в ней было выполнено
диагональное преобладание по сути эквивалентно задаче
поиска совершенного паросочетания в нашем графе.
Кажется, это что-то связанное с теоремой Холла о парасочетании
в двудольном графе... Доказательство теоремы дает нам алгоритм.
(За слова не отвечаю). Любопытный читатель может ознакомиться с реализацией
алгоритма в файле \href{https://github.com/vityaman-edu/comp-math-systems-of-linear-equations/blob/trunk/app/sle/include/method/iteration/diagonal-predominance.h}{diagonal-predominance.h}

\lstinputlisting[
    language={C++},
    caption={Попытка приведения СЛУ к диагональному преобладанию},
    linerange={33-91}
]{../app/sle/include/method/iteration/valid-sle.h}

\section{Реализация метода итераций на языке C++}

\lstinputlisting[
    language={C++},
    caption={Реализация метода итераций на языке C++},
    linerange={51-70}
]{../app/sle/include/method/iteration/solution.h}

\section{Примеры работы}
\subsection{Чтение из файла}

\lstinputlisting[
    caption={Входные данные из файла 1}
]{../res/1.txt}

\begin{lstlisting}[caption={Вывод программы 1}]
result.value = { 1.00001, 1.00001, 1.00002 }
result.error = { 4.48823e-05, 5.6386e-05, 7.11679e-05 }
result.steps_count = 9
\end{lstlisting}

\subsection{Чтение из консоли}

\begin{lstlisting}[caption={Использование программы из консоли}]
$ bazel-bin/app/main/app
0.1 3
2 2 10 14
10 1 1 12
2 10 1 13
result.value = { 1.01539, 1.01419, 1.02278 }
result.error = { 0.0493588, 0.0941295, 0.0967236 }
result.steps_count = 4
\end{lstlisting}


\section{Вывод}

Метод итераций оказался достаточно эффективным
для решения СЛУ. Кажется, что он позволяет
добиться достаточной точности за относительно
небольшое количество итераций. Несмотря
на то, что сам алгоритм дает лишь
приблизительный ответ, он не уступает прямым
в точности, ведь позволяет ее контролировать,
а в жизни нам часто достаточно лишь примерного
ответа до нескольких знаков после запятой.


\begin{thebibliography}{9}

    \bibitem{DemidovichMaron1966ru}
    Демидович Б. П., Марон И. А.
    Основы вычислительной математики /
    Демидович Б. П., Марон И. А. -
    М.: Наука, 1966. -
    294-302, 301 c.


    \bibitem{DvkLinal2023Itmo}
    Лекции по Линейной алгебре в университете ИТМО 
    2022 - 2023 гг / Карпов Д. В. \\
    \url{https://logic.pdmi.ras.ru/~dvk/ITMO/Algebra/2022-23/5_Lin_space.pdf}

    \bibitem{DvkLinal2022Itmo}
    Лекции по Теории графов в университете ИТМО
    2021 - 2022 гг / Карпов Д. В. \\
    \url{https://logic.pdmi.ras.ru/~dvk/ITMO/DM/2021-22/3_matchings.pdf}
    
\end{thebibliography}

\end{document}
