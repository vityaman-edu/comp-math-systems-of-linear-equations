\documentclass{article}

\usepackage[utf8]{inputenc}
\usepackage[russian]{babel}
\usepackage[a4paper, margin=1in]{geometry}
\usepackage{graphicx}
\usepackage{amsmath}
\usepackage{wrapfig}
\usepackage{multirow}
\usepackage{mathtools}
\usepackage{pgfplots}
\usepackage{pgfplotstable}
\usepackage{setspace}
\usepackage{changepage}
\usepackage{caption}
\usepackage{csquotes}
\usepackage{hyperref}

\pgfplotsset{compat=1.18}
\hypersetup{
  colorlinks = true,
  linkcolor  = blue,
  filecolor  = magenta,      
  urlcolor   = darkgray,
  pdftitle   = {
    comp-math-lab-1-report-systems-of-linear-equations-smirnov-victor
  },
}

\begin{document}



\begin{titlepage}
    \begin{center}
        \begin{spacing}{1.4}
            \large{Университет ИТМО} \\
        \end{spacing}
        \vfill
        \textbf{
            \huge{Лабораторная работа №1.} \\
            \huge{Решение системы линейных алгебраических уравнений}
        }
    \end{center}
    \vfill
    \begin{center}
        \begin{tabular}{r l}
            Группа:  & P32131                  \\
            Студент: & Смирнов Виктор Игоревич \\
            Вариант: & 2 (Метод итераций)      \\
        \end{tabular}
    \end{center}
    \vfill
    \begin{center}
        \begin{large}
            2023
        \end{large}
    \end{center}
\end{titlepage}

\section{Цель работы}
В новом семестре в рамках первой лабораторной работы
была поставлена задача реализовать так называемый метод
простых итераций для нахождения решения системы
линейных алгебраических уравнений (далее СЛУ).

\section{Задачи, решаемые при выполнении}
\begin{enumerate}
    \item Изучить метод итераций
    \item Разработать алгоритм приведения матрицы СЛУ
          эквивалентными преобразованиями к такому,
          в котором выполнено диагональное преобладание
    \item Разработать приложение для решения СЛУ
\end{enumerate}

\section{Описание метода итераций для решения СЛУ}
Сам алгоритм, условия сходимости и прочее прекрасно изложены
в книге Демидовича Б. П. и Марона И. А.
- "Основы вычислительной математики" 1966 года издания с
294 до 302 страниц, так что читатель может и должен для
дальнейшнего понимания происходящего ознакомиться с ним
самостоятельно.

\section{Получение СЛУ с диагональным преобладанием}
Одним из достаточных условий сходимости метода приближений является
наличие свойства диагонального преобладания матрицы коэффициентов
(далее будем называть ее основной матрицей).

\begin{equation}
    \forall i \in [1..n] \sum_{i \neq j} |a_{i, j}| < |a_{i,i}|
\end{equation}

Для приведения основной матрицы к нужному нам виду, нам
нужно применить к ней некоторое количество элементарных
преобразований. Напомню, что элементарные преобразования не
меняют решений СЛУ, так что мы не "испортим" нашу систему ими.

Один из методов также был описан в
"Основах вычислительной математики" на странице 301.
Будем следовать его описанию.

Вообще, самым простым элементарным преобразваением является
перестановка строк местами, так что давайте начнем с него.
Чего мы можем добиться, используя только это преобразование?
Можем максимально заполнить строки матрицы подходящими.

Пусть $P$ - множество позиций строк желаемой матрицы с
диагональным преобладанием, а $R$ - множество
строк изначальной матрицы.
Рассмотрим двудольный граф $G = (P, R, E): (p, r) \in E
    \Leftrightarrow ($для строки $r$ выполняется диагональное
    преобаладание на позиции $p)$. Заметим, что задача перестановки
строк в нашей матрице таким образом, чтобы в ней было выполнено
диагональное преобладание по сути эквивалентно задаче
поиска совершенного паросочетания в нашем графе.
Кажется, это что-то связанное с теоремой Холла о парасочетании
в двудольном графе... Доказательство теоремы дает нам алгоритм.
(За слова не отвечаю). Мне лень описывать алгоритм, но вы можете
посмотреть его реализацию в файле \href{https://github.com/vityaman-edu/comp-math-systems-of-linear-equations/blob/trunk/app/sle/include/method/iteration/diagonal-predominance.h}{diagonal-predominance.h}


\begin{thebibliography}{9}

    \bibitem{DemidovichMaron1966ru}
    \href{
        https://ikfia.ysn.ru/wp-content/uploads/2018/01/DemidovichMaron1966ru.pdf
    }{
        Демидович Б. П., Марон И. А. - "Основы вычислительной математики"
    }

    \bibitem{DvkLinal2023Itmo}
    \href{
        https://logic.pdmi.ras.ru/~dvk/ITMO/Algebra/2022-23/5_Lin_space.pdf
    }{
        Карпов Д. В. - Лекции по Линейной алгебре в университете ИТМО
        2022 - 2023 гг.
    }

    \bibitem{DvkLinal2022Itmo}
    \href{
        https://logic.pdmi.ras.ru/~dvk/ITMO/DM/2021-22/3_matchings.pdf
    }{
        Карпов Д. В. - Лекции по Теории графов в университете ИТМО
        2021 - 2022 гг.
    }

\end{thebibliography}

\end{document}
