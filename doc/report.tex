\documentclass{article}

\usepackage[utf8]{inputenc}
\usepackage[russian]{babel}
\usepackage[a4paper, margin=1in]{geometry}
\usepackage{graphicx}
\usepackage{amsmath}
\usepackage{wrapfig}
\usepackage{multirow}
\usepackage{mathtools}
\usepackage{pgfplots}
\usepackage{pgfplotstable}
\usepackage{setspace}
\usepackage{changepage}
\usepackage{caption}
\usepackage{csquotes}
\usepackage{hyperref}
\usepackage{listings}

\pgfplotsset{compat=1.18}
\hypersetup{
  colorlinks = true,
  linkcolor  = blue,
  filecolor  = magenta,      
  urlcolor   = darkgray,
  pdftitle   = {
    comp-math-lab-1-report-systems-of-linear-equations-smirnov-victor
  },
}

\definecolor{codegreen}{rgb}{0,0.6,0}
\definecolor{codegray}{rgb}{0.5,0.5,0.5}
\definecolor{codepurple}{rgb}{0.58,0,0.82}
\definecolor{backcolour}{rgb}{0.99,0.99,0.99}

\lstdefinestyle{mystyle}{
  backgroundcolor=\color{backcolour},   
  commentstyle=\color{codegreen},
  keywordstyle=\color{magenta},
  numberstyle=\tiny\color{codegray},
  stringstyle=\color{codepurple},
  basicstyle=\ttfamily\footnotesize,
  breakatwhitespace=false,         
  breaklines=true,                 
  captionpos=b,                    
  keepspaces=true,                 
  numbers=left,                    
  numbersep=5pt,                  
  showspaces=false,                
  showstringspaces=false,
  showtabs=false,                  
  tabsize=2
}

\lstset{style=mystyle}

\begin{document}

\begin{titlepage}
    \begin{center}
        \begin{spacing}{1.4}
            \large{Университет ИТМО} \\
        \end{spacing}
        \vfill
        \textbf{
            \huge{Лабораторная работа №1.} \\
            \huge{Решение системы линейных алгебраических уравнений}
        }
    \end{center}
    \vfill
    \begin{center}
        \begin{tabular}{r l}
            Группа:  & P32131                               \\
            Студент: & Смирнов Виктор Игоревич              \\
            Вариант: & 2, 17 (Метод итераций, Метод Гаусса) \\
        \end{tabular}
    \end{center}
    \vfill
    \begin{center}
        \begin{large}
            2023
        \end{large}
    \end{center}
\end{titlepage}

\section{Цель работы}
В новом семестре в рамках первой лабораторной работы
была поставлена задача реализовать так называемый метод
простых итераций для нахождения решения системы
линейных алгебраических уравнений (далее СЛУ).

\section{Задачи, решаемые при выполнении}
\begin{enumerate}
    \item Изучить метод итераций
    \item Разработать алгоритм приведения матрицы СЛУ
          эквивалентными преобразованиями к такому,
          в котором выполнено диагональное преобладание
    \item Разработать приложение для решения СЛУ
\end{enumerate}

\section{Описание метода итераций для решения СЛУ}
Сам алгоритм, условия сходимости и прочее прекрасно изложены
в книге Демидовича Б. П. и Марона И. А.
- "Основы вычислительной математики" 1966 года издания с
294 до 302 страниц, так что читатель может и должен для
дальнейшнего понимания происходящего ознакомиться с ним
самостоятельно.

\section{Получение СЛУ с диагональным преобладанием}
Одним из достаточных условий сходимости метода приближений является
наличие свойства диагонального преобладания матрицы коэффициентов
(далее будем называть ее основной матрицей).

\begin{equation}
    \forall i \in [1..n] \sum_{i \neq j} |a_{i, j}| < |a_{i,i}|
\end{equation}

Для приведения основной матрицы к нужному нам виду, нам
нужно применить к ней некоторое количество элементарных
преобразований. Напомню, что элементарные преобразования не
меняют решений СЛУ, так что мы не "испортим" нашу систему ими.

Один из методов также был описан в
"Основах вычислительной математики" на странице 301.
Будем следовать его описанию.

Вообще, самым простым элементарным преобразваением является
перестановка строк местами, так что давайте начнем с него.
Чего мы можем добиться, используя только это преобразование?
Можем максимально заполнить строки матрицы подходящими.

Пусть $P$ - множество позиций строк желаемой матрицы с
диагональным преобладанием, а $R$ - множество
строк изначальной матрицы.
Рассмотрим двудольный граф
$G = (P, R, E): (p, r) \in E \Leftrightarrow
    ($для строки $r$ выполняется диагональное
    преобаладание на позиции $p)
$. Заметим, что задача перестановки строк в нашей матрице таким
образом, чтобы в ней было выполнено диагональное преобладание по
сути эквивалентно задаче поиска совершенного паросочетания в
нашем графе. Кажется, это что-то связанное с теоремой Холла о
парасочетании в двудольном графе... Доказательство теоремы дает
нам алгоритм. (За слова не отвечаю). Мне лень описывать алгоритм,
но вы можете ознакомиться с ним в файле
\href{https://github.com/vityaman-edu/comp-math-systems-of-linear-equations/blob/trunk/src/math/sle/method/iteration/valid-sle.h#L39}{valid-sle.h}.

Реализация метода итераций по проверенной матрице
представлена ниже.

\lstinputlisting[
  language={C++}, 
  caption={Реализация метода итераций},
  linerange={63-86}
]{../src/math/sle/method/iteration/solution.h}  

\section{Решение СЛУ методом Гаусса}

Ситуация аналогичная с методом итераций:
сам алгоритм прекрасно описан в той же книжке,
в той же главе, так что я не буду вдаваться в 
подробности, а просто покажу вам код.

\lstinputlisting[
  language={C++}, 
  caption={Реализация метода Гаусса с выбором главного элемента},
  linerange={64-128}
]{../src/math/sle/method/gauss/solution.h}  

Основная идея алгоритма заключается в том, чтобы
свести решение системы уравнений $N \times N$ к 
системе уравнений $(N - 1) \times (N - 1)$. Таким
образом засчет конечности системы прийти к
системе из 1 уравнения и одного неизвестного, 
то есть тривиальному случаю. И, обладая навыком
решать систему уравнений $N \times N$, зная значения 
$N - 1$ переменных, можем по индукции решить всю 
задачу целиком. Попутно мы также считаем 
определитель матрицы и вектор неувязок.

\section{Примеры работы программы}

\subsection{Когда все хорошо}
\begin{lstlisting}
[fin]
0.01 3 
2  2  10 14
2  10  1 13
10 1   1 12

[stdin]
./app ../res/1.txt

[stdout]
Solving using gauss method...
result.value =  { 1.000, 1.000, 1.000 }
result.error =  { -0.000, 0.000, 0.000 }
|result.det| = -946.000
result.(a|b) = 
0.200   0.200   1.000   | 1.400
0.184   1.000   0.000   | 1.184
1.000   0.000   0.000   | 1.000

Solving using iteration method...
result.value =  { 1.000, 0.999, 0.999 }
result.error =  { 0.002, 0.002, 0.003 }
result.steps_count = 5

\end{lstlisting}

\subsection{Когда все плохо}
\begin{lstlisting}
[fin]
0.01 5
1 0 0 0 0 1
0 1 0 0 0 1
0 0 1 0 0 1
0 0 0 1 0 1
0 0 0 0 0 0

[stdin]
./app ../res/6.txt

[stdout]
Solving using gauss method...
result.value =  { -nan, -nan, -nan, -nan, -nan }
result.error =  { -nan, -nan, -nan, -nan, -nan }
|result.det| = 0.000
result.(a|b) = 
1.000   0.000   0.000   0.000   0.000   | 1.000
0.000   1.000   0.000   0.000   0.000   | 1.000
0.000   0.000   1.000   0.000   0.000   | 1.000
0.000   0.000   0.000   1.000   0.000   | 1.000
0.000   0.000   0.000   0.000   0.000   | 0.000

Solving using iteration method...
error: matrix can't be made diagonal predominant
\end{lstlisting}

\subsection{Когда метод итераций не вывозит}
\begin{lstlisting}
[fin]
0.01 4
7.4 2.2 -3.1  0.7 10
1.6 4.8 -8.5  4.5 10
4.7 7.0 -6.0  6.6 10
5.9 2.7  4.9 -5.3 10

[stdin]
./app ../res/2.txt

[stdout]
Solving using gauss method...
result.value =  { 1.412, -1.306, -1.316, 0.657 }
result.error =  { -3.412, -0.016, 1.973, -2.206 }
|result.det| = -1483.600
result.(a|b) = 
0.000   1.000   0.000   -0.352  | 1.875
-0.188  -0.565  1.000   -0.529  | -1.176
0.000   0.000   0.000   1.000   | -1.316
1.000   0.801   0.000   -0.397  | 2.311

Solving using iteration method...
error: matrix can't be made diagonal predominant
\end{lstlisting}

\begin{thebibliography}{9}

    \bibitem{DemidovichMaron1966ru}
    \href{
        https://ikfia.ysn.ru/wp-content/uploads/2018/01/DemidovichMaron1966ru.pdf
    }{
        Демидович Б. П., Марон И. А. - "Основы вычислительной математики"
    }

    \bibitem{DvkLinal2023Itmo}
    \href{
        https://logic.pdmi.ras.ru/~dvk/ITMO/Algebra/2022-23/5_Lin_space.pdf
    }{
        Карпов Д. В. - Лекции по Линейной алгебре в университете ИТМО
        2022 - 2023 гг.
    }

    \bibitem{DvkLinal2022Itmo}
    \href{
        https://logic.pdmi.ras.ru/~dvk/ITMO/DM/2021-22/3_matchings.pdf
    }{
        Карпов Д. В. - Лекции по Теории графов в университете ИТМО
        2021 - 2022 гг.
    }

\end{thebibliography}

\end{document}
